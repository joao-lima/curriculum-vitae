\documentclass[11pt,a4paper]{article} 
\usepackage[utf8]{inputenc}  
\usepackage{url}  
\usepackage{hyperref}
\usepackage[T1]{fontenc}
\usepackage{times}
\usepackage[margin=3cm]{geometry}
\usepackage[super]{nth}

\begin{document}

%%%%%%%%%%%%%%%%%%%%%%%%%%%%%%%%%%%%%%%%%%%%%%%%%%%%%%%%%%%%%%%%%%%%%%%%%%%%%%%
\begin{center}
{\sc\Large Curriculum Vit\ae}\\
\vspace{2mm}
{\bf\large João Vicente Ferreira Lima}\\
\vspace{2mm}
Associate Professor\\
%Universidade Federal de Santa Maria\\
Federal University of Santa Maria\\
Email: jvlima at inf.ufsm.br\\
Web: {\small \url{http://www.inf.ufsm.br/~jvlima}} \\
ORCID: {\small \url{https://orcid.org/0000-0002-2670-6963}}
%Google scholar: {\small \url{https://scholar.google.com.br/citations?user=jb6bKmoAAAAJ}}
\end{center}

%%%%%%%%%%%%%%%%%%%%%%%%%%%%%%%%%%%%%%%%%%%%%%%%%%%%%%%%%%%%%%%%%%%%%%%%%%%%%%%
\section{Address}
%%%%%%%%%%%%%%%%%%%%%%%%%%%%%%%%%%%%%%%%%%%%%%%%%%%%%%%%%%%%%%%%%%%%%%%%%%%%%%%

\begin{flushleft}
Universidade Federal de Santa Maria (UFSM)\\
DLSC - Centro de Tecnologia (CT)\\
Prédio 07, Anexo B - Sala 374 \\
Avenida Roraima 1000\\
Bairro Camobi\\
97105-900 Santa Maria - RS\\
Brazil
\end{flushleft}
 
%%%%%%%%%%%%%%%%%%%%%%%%%%%%%%%%%%%%%%%%%%%%%%%%%%%%%%%%%%%%%%%%%%%%%%%%%%%%%%%
\section{Education}
%%%%%%%%%%%%%%%%%%%%%%%%%%%%%%%%%%%%%%%%%%%%%%%%%%%%%%%%%%%%%%%%%%%%%%%%%%%%%%%

\begin{description}
\item[(2010--2014)] Ph.D., Computer Science, \emph{cotutelle} agreement between:
	\begin{itemize}
	\item Federal University of Rio Grande do Sul (UFRGS), Brazil.
	\item Grenoble University, Grenoble, France.
	\end{itemize}
\item[(2007--2009)] M.S, Computer Science, Federal University of Rio Grande do Sul (UFRGS) , Brazil.
\item[(2003--2007)] B.S, Computer Science, Federal University of Santa Maria (UFSM), Brazil.
\end{description}

%%%%%%%%%%%%%%%%%%%%%%%%%%%%%%%%%%%%%%%%%%%%%%%%%%%%%%%%%%%%%%%%%%%%%%%%%%%%%%%
\section{Areas of Interest} 
%%%%%%%%%%%%%%%%%%%%%%%%%%%%%%%%%%%%%%%%%%%%%%%%%%%%%%%%%%%%%%%%%%%%%%%%%%%%%%%

High performance computing, accelerators, parallel programming, energy efficiency, cluster computing, network security, big data.

%%%%%%%%%%%%%%%%%%%%%%%%%%%%%%%%%%%%%%%%%%%%%%%%%%%%%%%%%%%%%%%%%%%%%%%%%%%%%%%
\section{Experience}
%%%%%%%%%%%%%%%%%%%%%%%%%%%%%%%%%%%%%%%%%%%%%%%%%%%%%%%%%%%%%%%%%%%%%%%%%%%%%%%

%%%%%%%%%%%%%%%%%%%%%%%%%%%%%%%%%%%%%%%%%%%%%%%%%%%%%%%%%%%%%%%%%%%%%%%%%%%%%%%
\subsection{Teaching}

\begin{itemize}  \itemsep -2pt % reduce space between items
\item (07/2014 -- current) Associate Professor at the Federal University
of Santa Maria (UFSM), Santa Maria, RS, Brazil.
\end{itemize}

%%%%%%%%%%%%%%%%%%%%%%%%%%%%%%%%%%%%%%%%%%%%%%%%%%%%%%%%%%%%%%%%%%%%%%%%%%%%%%%
\subsection{Research}

\begin{itemize}  \itemsep -2pt % reduce space between items
\item (3/2015 -- current) Advisor at the Graduate Program in Computer
Science at UFSM (master course).

\item (03/2012 -- 02/2013) Research at Laboratoire d'Informatique de
Grenoble, Grenoble University, France, supported by CAPES/Brazil scholarship.

\item (03/2010 -- 02/2011) Research at Instituto de Informática, UFRGS, Brazil,
supported by CNPq/Brazil scholarship.

\item (09/2010 -- 02/2011) Research at Laboratoire d'Informatique de Grenoble,
Grenoble University, France, supported by Erasmus Mundus EBWII scholarship.

\item (04/2009 -- 02/2010) Research at Instituto de Informática, UFRGS, Brazil,
supported by CNPq/Brazil scholarship in project {\it Massive Atmosphere}.

\item (03/2007 -- 03/2009) Research at Instituto de Informática, UFRGS, Brazil,
supported by CAPES/Brazil scholarship, working on granularity of MPI-2 dynamic 
programs with processes and threads at runtime.
\end{itemize}

%%%%%%%%%%%%%%%%%%%%%%%%%%%%%%%%%%%%%%%%%%%%%%%%%%%%%%%%%%%%%%%%%%%%%%%%%%%%%%%
\subsection{System Administrator} 

\begin{itemize}  \itemsep -2pt % reduce space between items
\item (08/2005 -- 02/2007) System administrator at Núcleo de Ciência da
Computação (NCC), UFSM, suported by PRAE/UFSM and CPD/UFSM.
\end{itemize}

%%%%%%%%%%%%%%%%%%%%%%%%%%%%%%%%%%%%%%%%%%%%%%%%%%%%%%%%%%%%%%%%%%%%%%%%%%%%%%%
\subsection{Employment}

\begin{itemize}  \itemsep -2pt % reduce space between items
\item (09/2003--07/2005) CPD/UFSM SIE application and report development in Delphi.
\end{itemize}

%%%%%%%%%%%%%%%%%%%%%%%%%%%%%%%%%%%%%%%%%%%%%%%%%%%%%%%%%%%%%%%%%%%%%%%%%%%%%%%
\section{Awards}
%%%%%%%%%%%%%%%%%%%%%%%%%%%%%%%%%%%%%%%%%%%%%%%%%%%%%%%%%%%%%%%%%%%%%%%%%%%%%%%

\begin{itemize} \itemsep -2pt
\item First Place at the 6th Marathon of Parallel Programming - SBAC-PAD 2011 (SBC), Brazil. 
\item First Place at the 2nd GPU Programming Contest - SBAC-PAD 2011 (SBC), Brazil. 
\item Erasmus Mundus Euro Brazilian Windows II (EBWII) scholarship for 6 months
(09/2010--02/2011), European Commission.
\end{itemize}

%%%%%%%%%%%%%%%%%%%%%%%%%%%%%%%%%%%%%%%%%%%%%%%%%%%%%%%%%%%%%%%%%%%%%%%%%%%%%%%
\section{Teaching}
%%%%%%%%%%%%%%%%%%%%%%%%%%%%%%%%%%%%%%%%%%%%%%%%%%%%%%%%%%%%%%%%%%%%%%%%%%%%%%%

I teach mainly programming lectures at the undergraduate course of Computer
Science at UFSM since 2014.
\begin{itemize}
% 2019/2
\item ELC1067 - Laboratório de Programação II, \nth{2} semester 2019, 4 hours peer week.
\item ELC1016 - Sistemas Operacionais, \nth{2} semester 2019, 4 hours peer week.
% 2019/1
	\item ELC1035 - Prática em Sistemas Operacionais (\emph{Operating System Practice}), \nth{1} semester 2019, 4 hours peer week.
	\item ELC106 - Algoritmo e Programação, \nth{1} semester 2019, 4 hours peer week.
% 2018/1
\item ELC1067 - Laboratório de Programação II, \nth{2} semester 2018, 4 hours peer week.
\item ELC1016 - Sistemas Operacionais, \nth{2} semester 2018, 4 hours peer week.
%
\item DLSC801 - Computational Science, \nth{1} semester 2018, 4 hours peer week.
\item ELC1068 - Pesquisa e Ordenação de Dados ``A'', \nth{1} semester 2018, 4 hours peer week. 
%
\item ELC1067 - Laboratório de Programação II, \nth{2} semester 2017, 4 hours peer week.
\item ELC1035 - Prática em Sistemas Operacionais (\emph{Operating System Practice}), \nth{2} semester 2017, 4 hours peer week.
%
\item ELC1068 - Pesquisa e Ordenação de Dados ``A'', \nth{1} semester 2017, 4 hours peer week. 
\item ELC106 - Algoritmo e Programação, \nth{1} semester 2017, 4 hours peer week.
%
\item ELC1035 - Prática em Sistemas Operacionais (\emph{Operating System Practice}), \nth{2} semester 2016, 4 hours peer week.
\item ELC1067 - Laboratório de Programação II, \nth{2} semester 2016, 4 hours peer week.
\item ELC1067 - Laboratório de Programação II, \nth{1} semester 2016, 4 hours peer week.
\item ELC106 - Lógica e Programação, \nth{1} semester 2016, 4 hours peer week.
\item ELC1066 - Estruturas de Dados ``A'', \nth{2} semester 2015, 4 hours peer week.
\item ELC106 - Lógica e Programação, \nth{2} semester 2015, 4 hours peer week.
\item ELC1068 - Pesquisa e Ordenação de Dados ``A'', \nth{1} semester 2015, 4 hours peer week. 
\item ELC1067 - Laboratório de Programação II, \nth{1} semester 2015, 4 hours peer week.
\item ELC1067 - Laboratório de Programação II, \nth{2} semester 2014, 4 hours peer week.
\item ELC137 - Sistemas de Informação Distribuídos, \nth{2} semester 2014, 4 hours peer week.
\end{itemize}

%%%%%%%%%%%%%%%%%%%%%%%%%%%%%%%%%%%%%%%%%%%%%%%%%%%%%%%%%%%%%%%%%%%%%%%%%%%%%%%
\section{Supervision}
%%%%%%%%%%%%%%%%%%%%%%%%%%%%%%%%%%%%%%%%%%%%%%%%%%%%%%%%%%%%%%%%%%%%%%%%%%%%%%%

%%%%%%%%%%%%%%%%%%%%%%%%%%%%%%%%%%%%%%%%%%%%%%%%%%%%%%%%%%%%%%%%%%%%%%%%%%%%%%%
\subsection{Master students}

\begin{itemize} \itemsep -2pt
\item Lucas Ferreira da Silva (2019--): Big Data Processing with Low Power Devices.
\item Rafael Gauna Trindade (2018--): Parallel Adaptive Loop Algorithms for Asymetric Multicore Processors.
\item Alexsander Haas (2017--2019): A Big Data System for Network Traffic Analysis.
\item Gabriel Freytag (2016--2018): A Data-Flow Task-based Implementation of the 
Lattice-Boltzmann Method.
\item Daniel Di Domenico (2015--2017): HPSM: A C++ API for Parallel Loop Programs Supporting Multi-CPUs and Multi-GPUs.
\end{itemize}

%%%%%%%%%%%%%%%%%%%%%%%%%%%%%%%%%%%%%%%%%%%%%%%%%%%%%%%%%%%%%%%%%%%%%%%%%%%%%%%
\subsection{Undergraduate students}
\begin{itemize} \itemsep -2pt
\item Andre Rakowski (2018--): Signal Processing of IoT Devices on Railway Systems.
\item Rafael Gauna Trindade (2016--2017): C++ Programming Interfaces for
  Scientific Applications.
\item Pedro Langbecker Lima (2015--2016): Scientific Applictions using
  OpenMP 4.
\end{itemize}

%%%%%%%%%%%%%%%%%%%%%%%%%%%%%%%%%%%%%%%%%%%%%%%%%%%%%%%%%%%%%%%%%%%%%%%%%%%%%%%
\section{Software}
%%%%%%%%%%%%%%%%%%%%%%%%%%%%%%%%%%%%%%%%%%%%%%%%%%%%%%%%%%%%%%%%%%%%%%%%%%%%%%%

\begin{enumerate}
\item {\bf GitHub} Web site: \url{https://github.com/joao-lima}.
\item {\bf XKaapi} I have been involved in the development of 
the XKaapi runtime system since 2010. Web site: \url{http://kaapi.gforge.inria.fr}.
\end{enumerate}

%%%%%%%%%%%%%%%%%%%%%%%%%%%%%%%%%%%%%%%%%%%%%%%%%%%%%%%%%%%%%%%%%%%%%%%%%%%%%%%
\section{Publications}
%%%%%%%%%%%%%%%%%%%%%%%%%%%%%%%%%%%%%%%%%%%%%%%%%%%%%%%%%%%%%%%%%%%%%%%%%%%%%%%

Google scholar link: {\small \url{https://scholar.google.com.br/citations?user=jb6bKmoAAAAJ}}
%%%%%%%%%%%%%%%%%%%%%%%%%%%%%%%%%%%%%%%%%%%%%%%%%%%%%%%%%%%%%%%%%%%%%%%%%%%%%%%
%\subsection{Book chapters}

%%%%%%%%%%%%%%%%%%%%%%%%%%%%%%%%%%%%%%%%%%%%%%%%%%%%%%%%%%%%%%%%%%%%%%%%%%%%%%%
\subsection{International peer-reviewed journal}

\begin{itemize} \itemsep -2pt

\item
João~V.~F. Lima, Daniel Di Domenico.
``HPSM: A Programming Framework to Exploit Multi-CPU and Multi-GPU Systems Simultaneously''.
\emph{International Journal of Grid and Utility Computing}, v. 10, p. 201-211, 2019. 

\item
João V. F. Lima, Issam Ra\"{i}s, Laurent Lefèvre, and Thierry Gautier.
``Performance and energy analysis of OpenMP runtime systems with dense linear algebra algorithms''.
\emph{International Journal of High Performance Computing Applications}, v. 33, p. 431-443, 2018.

\item
Daniel Di Domenico, João~V.~F. Lima, Andrea S. Charão. 
``OpenMP with parallel loops or asynchronous tasks: a performance evaluation focusing the NQueens benchmark''.
\emph{IEEE Latin America Transactions}, v. 15, p. 1793-1800, 2017.

\item 
João~V.~F. Lima, Thierry Gautier, Vincent Danjean, Bruno Raffin, and Nicolas Maillard.
``Design and Analysis of Scheduling Strategies for Multi-CPU and Multi-GPU Architectures''.
\emph{Parallel Computing}, p. 37-52, 2015.


\item João V. F. Lima, Nicolas Maillard.
``Online mapping of MPI-2 dynamic tasks to processes and threads''.
\emph{International Journal of High Performance Systems Architecture (IJHPSA)}, v. 2, pp. 81-89, 2009.
%doi: \href{http://dx.doi.org/10.1504/IJHPSA.2009.032025}{10.1504/IJHPSA.2009.032025}.
\end{itemize}

%%%%%%%%%%%%%%%%%%%%%%%%%%%%%%%%%%%%%%%%%%%%%%%%%%%%%%%%%%%%%%%%%%%%%%%%%%%%%%%
\subsection{International peer-reviewed conference proceedings}

\begin{itemize} \itemsep -2pt

\item
Gabriel Freytag, Matheus S. Serpa, João~V.~F. Lima, Paolo Rech, Philippe O. A. Navaux.
``Non-Uniform Partitioning for Collaborative Execution on Heterogeneous Architectures''.
\emph{31th International Symposium on Computer Architecture and
High Performance Computing (SBAC-PAD)},
Campo Grande, Brazil, 2019.
(Accepted for publication)
	
\item
João~V.~F. Lima, Gabriel Freytag, Vinicius G. Pinto, Claudio Schepke, Philippe O. A. Navaux.
`` A Dynamic Task-Based D3Q19 Lattice-Boltzmann Method for Heterogeneous Architectures''.
\emph{27th Euromicro International Conference on Parallel, Distributed and NetworkBased Processing (PDP)},
Pavia, Italy, 2019.

\item
Gabriel Freytag, Philippe O. A. Navaux, João~V.~F. Lima, Lucas M. Schnorr, Paolo Rech.
``Non-Uniform Domain Decomposition of the Lattice-Boltzmann Method for Heterogeneous Accelerated Processing Units''.
\emph{3th International Meeting on High Performance Computing for Computational Science (VECPAR 2018)}, 
São Pedro, SP, Brazil, 2018.

\item
Rafael G. Trindade, João~V.~F. Lima, Andrea S. Charão.
``Performance Evaluation of Deep Learning Frameworks over Different Architectures''.
\emph{3th International Meeting on High Performance Computing for Computational Science (VECPAR 2018)}, 
São Pedro, SP, Brazil, 2018.

\item
Rapha\"el Bleuse, Thierry Gautier, João V. F. Lima, Gregory Mounie, and Denis Trystram.
``Scheduling data flow program in XKaapi: A new affinity-based algorithm for heterogeneous architectures''.
\emph{Proc. of the 20th Euro-Par}, 2014, Porto, Portugal.

\item 
João~V.~F. Lima, Fran\c{c}ois Broquedis, Thierry Gautier, and Bruno Raffin.
``Preliminary Experiments with XKaapi on Intel Xeon Phi Coprocessor''.
\emph{25th International Symposium on Computer Architecture and
  High Performance Computing (SBAC-PAD)}, Porto de Galinhas, Brazil, 2013.
%\newblock doi: \href{http://dx.doi.org/10.1109/SBAC-PAD.2013.28}{10.1109/SBAC-PAD.2013.28}.

\item 
Thierry Gautier, João~V.~F. Lima, Nicolas Maillard, and Bruno Raffin.
``XKaapi: A Runtime System for Data-Flow Task Programming on Heterogeneous Architectures''.
\emph{2013 IEEE 27th International Symposium on Parallel Distributed Processing (IPDPS)}, p.~1299--1308, 2013.
%\newblock doi: \href{http://dx.doi.org/10.1109/IPDPS.2013.66}{10.1109/IPDPS.2013.66}.

\item
Thierry Gautier, João~V.~F. Lima, Nicolas Maillard, and Bruno Raffin.
``Locality-Aware Work Stealing on Multi-CPU and Multi-GPU Architectures''.
\emph{6th Workshop on Programmability Issues for Heterogeneous Multicores (MULTIPROG)}, 
p.~51--62, Berlin, Germany, 2013.

\item 
João~V.~F. Lima, Thierry Gautier, Nicolas Maillard, and Vincent Danjean.
``Exploiting Concurrent GPU Operations for Efficient Work Stealing on Multi-GPUs''.
\emph{24th International Symposium on Computer Architecture and High Performance Computing (SBAC-PAD)}, p.~75--82, New York, NY, USA, 2012.
%\newblock doi: \href{http://dx.doi.org/10.1109/SBAC-PAD.2012.28}{10.1109/SBAC-PAD.2012.28}.

\item
Marco~A.~Z.~Alves, Márcia~C.~Cera, João~V.~F. Lima, Nicolas Maillard, and Philippe O. A. Navaux.
``Enhancing Energy Efficiency using Efficient Parallel Programming Techniques''.
\emph{3o Conferencia Latino Americana de Computación de Alto Rendimiento (CLCAR 2010)},
p.~117--124, Gramado, Brazil, 2012.

\end{itemize}

%%%%%%%%%%%%%%%%%%%%%%%%%%%%%%%%%%%%%%%%%%%%%%%%%%%%%%%%%%%%%%%%%%%%%%%%%%%%%%%
%\subsection{National peer-reviewed conference proceedings}

%%%%%%%%%%%%%%%%%%%%%%%%%%%%%%%%%%%%%%%%%%%%%%%%%%%%%%%%%%%%%%%%%%%%%%%%%%%%%%%
\subsection{Other conference proceedings}

\begin{itemize} \itemsep -2pt

\item Márcia C. Cera, João V. F. Lima, Nicolas Maillard, and Philippe O. A. Navaux.
``Challenges and Issues of Supporting Task Parallelism in MPI''.
\emph{17th European MPI Users' Group Meeting (EuroMPI 2010)},
p.~302--305, Stuttgart, Germany, 2010.
%\newblock doi: \href{http://dx.doi.org/10.1007/978-3-642-15646-5_34}{10.1007/978-3-642-15646-5\_34}.

\end{itemize}

%%%%%%%%%%%%%%%%%%%%%%%%%%%%%%%%%%%%%%%%%%%%%%%%%%%%%%%%%%%%%%%%%%%%%%%%%%%%%%%
%\subsection{Tutorials}

%\begin{itemize} \itemsep -2pt
%\item um
%\item dois
%\end{itemize}

\end{document}
